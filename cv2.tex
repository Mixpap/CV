%%% LaTeX Template: Designer's CV
%%%
%%% Source: http://www.howtotex.com/
%%% Feel free to distribute this template, but please keep the referal to HowToTeX.com.
%%% Date: March 2012


%%%%%%%%%%%%%%%%%%%%%%%%%%%%%%%%%%%%%
% Document properties and packages
%%%%%%%%%%%%%%%%%%%%%%%%%%%%%%%%%%%%%
\documentclass[a4paper,12pt,final]{memoir}

\usepackage[usenames,dvipsnames]{xcolor}

\definecolor{primary}{HTML}{2b2b2b} % The primary document color for content text
\definecolor{headings}{HTML}{6A6A6A} % The color of the large sections
\definecolor{subheadings}{HTML}{333333} % The color of subsections and places worked/studied
\definecolor{date}{HTML}{666666} % The color used for the Last Updated text at the top right

\usepackage{todonotes}
\usepackage{xltxtra}
\usepackage{xunicode}
\usepackage{xgreek}
\usepackage{fontspec}
\setmainfont[Path=fonts/,BoldItalicFont=cfjelbi,BoldFont=cfjeb,ItalicFont=cfjeli]{cfjerg}
\setsansfont[Scale=MatchLowercase,Mapping=tex-text, Path = fonts/]{cfjerg}
% misc
%\renewcommand{\familydefault}{bch}	% font
\pagestyle{empty}					% no pagenumbering
\setlength{\parindent}{0pt}			% no paragraph indentation


% required packages (add your own)
\usepackage{flowfram}										% column layout
\usepackage[top=1cm,left=1cm,right=1cm,bottom=1cm]{geometry}% margins
\usepackage{graphicx}										% figures
\usepackage{url}											% URLs
\usepackage[usenames,dvipsnames]{xcolor}					% color
\usepackage{multicol}										% columns env.
	\setlength{\multicolsep}{0pt}
\usepackage{paralist}										% compact lists
\usepackage{tikz}



%%%%%%%%%%%%%%%%%%%%%%%%%%%%%%%%%%%%%
% Create column layout
%%%%%%%%%%%%%%%%%%%%%%%%%%%%%%%%%%%%%
% define length commands
\setlength{\vcolumnsep}{\baselineskip}
\setlength{\columnsep}{\vcolumnsep}

%% frame setup (flowfram package)
%% left frame
%\newflowframe{0.2\textwidth}{\textheight}{0pt}{0pt}[left]
%	\newlength{\LeftMainSep}
%	\setlength{\LeftMainSep}{0.2\textwidth}
%	\addtolength{\LeftMainSep}{2\columnsep}
%% right frame
%\newflowframe{0.7\textwidth}{\textheight}{\LeftMainSep}{0pt}[main01]
%
%% horizontal rule between frames (using TikZ)
%\renewcommand{\ffvrule}[3]{%
%\hfill
%\tikz{%
%	\draw[loosely dotted,color=RoyalBlue,line width=1.5pt,yshift=-#1] 
%	(0,0) -- (0pt,#3);}%
%\hfill\mbox{}}
%\insertvrule{flow}{1}{flow}{2}

% left frame
\newflowframe{0.2\textwidth}{\textheight}{0pt}{0pt}[left]
\newlength{\LeftMainSep}
\setlength{\LeftMainSep}{0.2\textwidth}
\addtolength{\LeftMainSep}{1\columnsep}

% small static frame for the vertical line
\newstaticframe{1.5pt}{\textheight}{\LeftMainSep}{0pt}

% content of the static frame
\begin{staticcontents}{1}
	\hfill
	\tikz{%
		\draw[loosely dotted,color=RoyalBlue,line width=1.5pt,yshift=0]
		(0,0) -- (0,\textheight);}%
	\hfill\mbox{}
\end{staticcontents}

% right frame
\addtolength{\LeftMainSep}{1.5pt}
\addtolength{\LeftMainSep}{1\columnsep}
\newflowframe{0.7\textwidth}{\textheight}{\LeftMainSep}{0pt}[main01]

%%%%%%%%%%%%%%%%%%%%%%%%%%%%%%%%%%%%%
% define macros (for convience)
%%%%%%%%%%%%%%%%%%%%%%%%%%%%%%%%%%%%%
\newcommand{\Sep}{\vspace{0.75em}}
\newcommand{\SmallSep}{\vspace{0.25em}}

\newenvironment{AboutMe}
	{\ignorespaces\textbf{\color{RoyalBlue} \textbullet{}}}
	%{\Sep\ignorespacesafterend}
	
\newcommand{\CVSection}[1]
	{\Large{#1}\par
	\SmallSep\normalsize\normalfont}


\newcommand{\CVItem}[2]
	{\textbf{\color{RoyalBlue} #1} #2}


\newcommand{\location}[1]{ % Used for specifying a duration and/or location under a subsection
\small{\color{headings}#1}} 

\newenvironment{tightitemize} % Defines the tightitemize environment which modifies the itemize environment to be more compact
{\vspace{-\topsep}\begin{itemize}\itemsep1pt \parskip0pt \parsep0pt}
{\end{itemize}\vspace{-\topsep}}

%%%%%%%%%%%%%%%%%%%%%%%%%%%%%%%%%%%%%
% Begin document
%%%%%%%%%%%%%%%%%%%%%%%%%%%%%%%%%%%%%
\begin{document}

% Left frame
%%%%%%%%%%%%%%%%%%%%
%\begin{figure}
%	\hfill
%	\includegraphics[width=0.6\columnwidth]{photo}
%	\vspace{-7cm}
%\end{figure}
\begin{flushright}
Βιογραφικό Σημείωμα
\vspace{0.75cm}

\small
	Παπαχρήστου Μιχάλης \\
	\url{mighalis@gmail.com}  \\
	\url{github.com/Mixpap} \\
	τηλ. 6936741032
\end{flushright}\normalsize
\framebreak


% Right frame
%%%%%%%%%%%%%%%%%%%%
\Huge {\color{RoyalBlue} Παπαχρήστου Μιχάλης} \\
%\Large\bfseries  Graphics designer \\

\normalsize\normalfont

% About me
\begin{AboutMe}
Γεννηθείς το 1987, μόνιμος κάτοικος Ελληνικού Αττικής.
\end{AboutMe}

% Experience
\CVSection{Εργασιακή Εμπειρία}

\SmallSep

\CVItem{Aπασχόληση σε πλήθος διαφορετικών εργασιακών χώρων}{\newline}
\location{Eστίαση 2007-2013, Tηλεφωνική Yποστήριξη και Πωλήσεις 2009-2010 κ.α.}


\CVItem{Ιδιαίτερα Μαθήματα}{}
\begin{tightitemize}
	\item Μαθητές Δευτεροβάθμιας εκπαίδευσης σε μαθηματικά, φυσική, χημεία και πληροφορική \location{2008-2016}
	\item Φοιτητές Τριτοβάθμιας εκπαίδευσης (προγραμματισμός)
	\location{2011-2012,2014}
\end{tightitemize}
\SmallSep

% Education
\CVSection{Εκπαίδευση}
\CVItem{Δίπλωμα Βασικής Γνώσης Αγγλικών}{\newline}
\location{First Certificate in English (Grade C) | 2002}
\SmallSep

\CVItem{Απολυτήριο Ενιαίου Λυκείου}{\newline}
\location{1ο Ενιαίο Λύκειο Ελληνικού με βαθμό 16.5 | 2005}
\SmallSep

\CVItem{Πτυχιακή εργασία στο Τμήμα Φυσικής: Τα φυσικά χαρακτηριστικά των μοριακών εκροών στο νεφέλωμα W3 \newline}
\SmallSep

\CVItem{Πτυχίο Φυσικής Σχολής Θετικών Επιστημών του Πανεπιστήμιο Αθηνών}{\newline}
\location{Με βαθμολογία "Λίαν Καλώς" | 2015}
\SmallSep

\CVItem{Μεταπτυχιακό πρόγραμμα Αστροφυσικής-Αστρονομίας-Μηχανικής στο τμήμα Φυσικής του Πανεπιστήμιο Αθηνών }{\newline}
\location{2015 -}
\SmallSep



% Skills
\CVSection{Δεξιότητες και λοιπές γνώσεις}
\CVItem{Προγραμματισμός και ανάπτυξη λογισμικού:}{}
\begin{tightitemize}
\item Μέση εμπειρία ανάπτυξης επιστημονικού και στατιστικού προγραμματισμού (Scientific-Statistical Computing), μεθόδους ανάλυσης-επεξεργασίας δεδομένων (Data Analysis) και βασικές μεθόδους Μηχανικής Μάθησης (Machine Learning) στις γλώσσες \textbf{Python} \textbullet{} \textbf{Julia} \textbullet{} \textbf{R} \textbullet{} \textbf{Matlab}
\item Βασικές γνώσεις ανάπτυξης λογισμικού σε: C \textbullet{} C++ \textbullet{} Java \textbullet{} Fortran \textbullet{} Assembly 
\item Εξοικείωση με λειτουργικά συστήματα \textbf{Linux} και \textbf{BSD}
\item Εξοικείωση με το σύστημα κατανεμημένου ελέγχου (distributed revision system) \textbf{Git} - \textbf{GitHub}
\end{tightitemize}

\CVItem{Γραφιστική και στοιχειοθεσία:}{}
\begin{tightitemize}
\item Εξοικείωση με το σύστημα στοιχειοθεσίας \LaTeX\
\item Βασικές γνώσεις γραφιστικής και εφαρμογή μέσω των πακέτων \textbf{Inkscape} \textbullet{} \textbf{Adobe Illustrator} \textbullet{} \textbf{Xara} \textbullet{} \textbf{Corel Draw}
\end{tightitemize}
\SmallSep

\CVSection{Ερευνητικά Ενδιαφέροντα}
\CVItem{Πρακτική Άσκηση}{}
\begin{tightitemize}
	\item «Εφαρμογή Τεχνητών Νευρωνικών Δικτύων (ANN) στην ταξινόμηση και παραμετροποίηση φασμάτων γαλαξιών» \\
	\location{Νοέμβρης-Δεκέμβρης 2014 | Ινστιτούτο Αστρονομίας, Αστροφυσικής, Διαστημικών Εφαρμογών \& Τηλεπισκόπησης, Εθνικό Αστεροσκοπείο Αθηνών | Επιβλέπων: Δρ. Ιωάννης Μπέλλας-Βελίδης}
\end{tightitemize}

%=========================================================
\clearpage
\framebreak
\framebreak
%=======================================================

\CVItem{Σχολεία - Workshops}{}
\begin{tightitemize}
	\item Statistical analysis of astronomical data sets \\
	\location{3-7 Ιουνίου 2013 | Πανεπιστήμιο Αθηνών, τμήμα Φυσικής}
	\item The Role and the Origin of Magnetic Fields in Astrophysics \\
	\location{11-12 Μαρτίου 2013 | Ακαδημία Αθηνών}
	\item The Unification Model of Active Galactic Nuclei \\
	\location{4-5 Δεκεμβρίου 2014 | 5ο Χειμερινό Σχολείο Αστροφυσικής του Εθνικού Αστεροσκοπείου Αθηνών | Πανεπιστήμιο Αθηνών, τμήμα Φυσικής}
\end{tightitemize}
\SmallSep


\CVItem{Παρακολούθηση Σεμιναρίων}{}
\begin{tightitemize}
\item Ομάδα Αστροφυσικής του Πανεπιστημίου Αθηνών \location{2010-2016}
\item Κέντρο Ερευνών Αστρονομίας και Εφαρμοσμένων Μαθηματικών της Ακαδημίας Αθηνών \location{2009-2012}
\item Εθνικό Αστεροσκοπείο Αθηνών (ΙΑΑΔΕΤ) \location{2014-2015}
%\item Σεμινάρια Θεωρίας Γραφημάτων στο τμήμα Μαθηματικών (Μεταπτυχιακό Πρόγραμμα Λογικής και Αλγορίθμων) \location{2012}
\item Σεμινάρια ανοιχτού λογισμικού στο τμήμα πληροφορικής του Πανεπιστημίου Πειραιά \location{2006}
\end{tightitemize}
\SmallSep


%%%%%%%%%%%%%%%%%%%%%%%%%%%%%%%%%%%%%
% End document
%%%%%%%%%%%%%%%%%%%%%%%%%%%%%%%%%%%%%
\end{document}